The chapter is based on the published article \mycite[j1]. The discussed video content is also included in article assets. The following paragraphs highlight the main findings and results achieved from the date of publication.

\sec Lightning duration

The duration of the lightning event was determined by identifying local maxima of illumination on the camera chip over time from peaks that had a prominence higher than twice the noise level. The first and last maxima were used to identify the first and last frame, and the lightning duration was calculated from the framerate. The first ten frames of the camera recording were omitted to avoid potential artifacts from video data compression. The noise amplitude was established from the first 100 frames (from frame 10 to frame 110). Examples of duration determination for some lightning events are shown in Figure \ref[Camera_lightning_duration]. The histogram of lightning event durations, shown in Figure \ref[duration_histogram], includes 107 lightning events measured during 12 thunderstorms.

\medskip
\clabel[Camera_lightning_duration]{Camera lightning events duration}
\picw=15cm \cinspic ./img/camera_lightning_duration.png
\caption/f   Examples of detection of lightning events duration. Red crosses are detected peaks in signals generated by lightning processes. The determined length of lightning duration is highlighted. The duration of the camera frame is 620~µs.
\medskip

To calculate the total length of a lightning event from a radio signal (see Fig. \ref[UHF_lightning_duration]), a similar methodology was used, where a signal greater than $4\sigma$, calculated from a moving window of 10~ms, marks the beginning of the lightning discharge. This algorithm is applied symmetrically from both ends of the recorded signal. For detecting the lightning start, the process begins from the first sample of the signal, and for detecting the lightning end, the process begins from the last sample of the recorded signal.


\medskip
\clabel[UHF_lightning_duration]{UHF lightning events duration}
\picw=15cm \cinspic ./img/QFH_lightning_duration.png
\caption/f    Example of lightning duration measurement based on radio signal captured by the UHF receiver. The time duration of lightning is marked by the green overlay.
\medskip


\medskip
\clabel[duration_histogram]{Histogram of durations of observed lightning events}
\picw=15cm \cinspic ./img/Duration_histogram.png
\caption/f   Histogram of the observed duration of lightning events. The histogram is based on data captured by high-speed cameras, but subsequent analysis shows that lightning duration observed by the camera is consistent also with VLF and UHF receivers.
\medskip

The measured durations are usually in the order of hundreds of milliseconds, only exceptionally there are shorter lightning events. The median duration of lightning events is 524 ms. This result is significantly higher than was mentioned in previous studies 200-300~ms \cite[rakov_uman_2003] and 350~ms  \cite[López_Pineda_Montanyà_Velde_Fabró_Romero_2017].

The processing of lightning recordings differs for day and night storm observations as they require different settings of high-speed cameras, in particular, the setting of exposition length and analog gain varies. This difference could affect the exact measurements of the total length of a discharge, resulting however only in a shortening of the estimated time duration. Because of sunlight scattered in the atmosphere during the daytime, the weak discharges might have been omitted. Therefore the extracted time durations of lightning events could be possibly underestimated.

From the subsequent analysis of the lightning duration, two thunderstorms were excluded where it was not possible to distinguish individual lightning events, meaning there was no delay of at least 500 ms between the individual detected discharges. Lightning is considered to be an event where the time between the individual discharges does not exceed 100 ms.

\sec[Lightning_characterization] Observed lightning processes

Data from the cameras reveal similar phases of lightning development. Examples of video recordings with individual phases visible and not obscured by clouds are provided.
Video \ref[video_1627302745.846055.mp4] captures (from T=+0.27 s (1:11) to T=+0.40 s (1:22)) a positive side of the leader. When the current flowing through the leader begins to weaken, recoil leaders start to appear at its end (T=+0.38 s (1:21)). The term recoil leader was taken from literature \cite[MAZUR2013763]. Based on the described observations, however, it is not confirmed that all the visible recoil leaders reuse an exactly already established ionized channel.

% static figure
%\medskip
%\clabel[video_1627302745.846055.mp4]{Video -- positive side of the leader}
%\picw=15cm \cinspic ./img/1627302745.846055.png
%\caption/f   The embedded video file: {\em 1627302745.846055.mp4} shows lightning together with the positive side of leaders and then recoil leaders. Please click on the Figure to play the video, alternatively, the file could be extracted from the PDF or replayed on the web \url{https://youtu.be/uvGOy2K-y8s}.
%\medskip


\midinsert
\clabel[video_1627302745.846055.mp4]{Video -- positive side of the leader}
\render[1627302745.846055.mp4][
name=bigvideo,
controls=true,
repeat=0,
]{\picwidth=\hsize \inspic{./img/1627302745.846055.png}}
\caption/f   The embedded video file: {\em 1627302745.846055.mp4} shows lightning together with the positive side of leaders and then recoil leaders. Please click on the Figure to play the video, alternatively, the file could be extracted from the PDF or replayed on the web \hbox{\url{https://youtu.be/uvGOy2K-y8s}}.
\endinsert

Video \ref[video_2021-08-15-20-07-35.912167-lightning.mp4] captures an invisible positive leader, only blurred recoil leaders are visible. From T=+0.07 s (1:36) a negative side of the leader is visible. Negative leader branches abundantly, its propagation is faster than the positive leader and contains hot luminous ends. The second visible negative leader starts at T=+0.25 s (1:51). The negative leaders do not generate recoil leaders.


%\medskip
%\clabel[video_2021-08-15-20-07-35.912167-lightning.mp4]{Video -- negative side of the leader}
%\picw=15cm \cinspic ./img/2021-08-15-20-07-35.912167-lightning.png
%\caption/f   The embedded video file: {\em 2021-08-15-20-07-35.912167-lightning.mp4} shows negative leader branching abundantly, its propagation is faster than the positive leader and contains hot luminous ends. Please click on the Figure to play the video, alternatively, the file could be extracted from the PDF or replayed on the web \url{https://youtu.be/DjffA6NS8dg}.
%\medskip


\midinsert
\clabel[video_2021-08-15-20-07-35.912167-lightning.mp4]{Video -- negative side of the leader}
\render[2021-08-15-20-07-35.912167-lightning.mp4][
name=bigvideo,
controls=true,
repeat=0,
]{\picwidth=\hsize \inspic{./img/2021-08-15-20-07-35.912167-lightning.png}}
\caption/f   The embedded video file: {\em 2021-08-15-20-07-35.912167-lightning.mp4} shows negative leader branching abundantly, its propagation is faster than the positive leader and contains hot luminous ends. Please click on the Figure to play the video, alternatively, the file could be extracted from the PDF or replayed on the web \hbox{\url{https://youtu.be/DjffA6NS8dg}}.
\endinsert

\secc[VLF_LMA] Novel method of VLF lightning mapping

The described findings have direct implications, for example, on the way lightning discharge events should be displayed and localized. Given the typical dimensions and time scales on which discharges occur, it follows that even receivers operating on the VLF principle can provide meaningful descriptions of lightning discharge structures. At the same time, for the same reason, it is misleading to display lightning discharges on maps as points, which are intuitively interpreted as CG lightning strikes hitting the ground at approximately the position of the map mark. In reality, most of those points correspond to CC lightning discharges, given the actual incidence of their occurrence over CG lightning.

A better representation that considers this fact is attempted in Fig. \ref[VLF_lightning_map], which shows a calculated approximate visualization of the lightning structure. The representation is based on VLF data obtained from only three measuring stations placed on cars, so it contains many simplifications. For example, the lightning structure is depicted in a 2D plane positioned at a fixed height above the ground. Additionally, it is important to note that lightning discharges do not radiate at branching points, but primarily at the locations of two-thirds recoil leaders. The drawn connections do not reflect this fact, as the depiction is currently an illustrative algorithm that, despite its limitations, is a closer description of the physical principle of lightning discharges than the widely used depiction of points in a map.

\medskip
\clabel[VLF_lightning_map]{Lightning map based on VLF data}
\picw=15cm \cinspic ./img/VLF_lightning_map.png
\caption/f   An illustrative visualization of the structure of lightning discharges based on VLF data obtained from three measuring stations mounted on vehicles. The representation depicts the lightning structure in a 2D plane at a fixed height above the ground, capturing the approximate spatial distribution and progression of the lightning channels. The depiction emphasizes the complexity of the lightning and provides a closer representation of the physical nature of lightning discharges.
\medskip

The algorithm used\urlnote{https://github.com/ODZ-UJF-AV-CR/CRREAT_cars/blob/master/VLF_lightning/VLF_location.ipynb} for processing and visualizing VLF recordings from three measurement vehicles operates as follows. First, the recordings are aligned based on absolute time marks. Next, the data fragments are normalized, and any offsets are removed. The prepared recordings are then divided into a predetermined number of segments. The segment length varies, chosen to ensure each segment contains a similar amount of energy. This energy level and segmentation are determined algorithmically based on the total energy of the weakest signal. This method ensures each segment has significant morphological features necessary for calculating signal shifts. Finally, TDoA values are calculated by cross-correlating pairs of signals from the three stations.

The localization algorithm utilizes segments with at least two TDoA values that match physical constraints (e.g., time shifts cannot exceed the physical distance between stations). The navigation solution is then obtained by numerically solving the TDoA localization problem within an area situated above the plane where the stations are placed, at the expected cloud base altitude.

The fragment signal processing method is selected due to the aforementioned results. Additionally, the use of cross-correlation over signal segments provides significant advantages when one of the signals (in this case, CAR1) is heavily interfered with by uncorrelated noise absent from other stations. That is a feature that cannot be applied on conventional lightning detection networks, where only individually recorded pulses are processed.

The visualization is achieved through a graph algorithm that progressively adds points to the graph in a way that satisfies the geometric conditions related to lightning propagation, like utilizing adjacent lightning channels. This approach effectively eliminates the need for clustering algorithms commonly used \cite[Sibolla2021] to determine the lightning strike area, as the connection between individual parts of lightning is not lost in the process. The described algorithm could be significantly improved by implementing an algorithmic estimation of the optimal number of segments based on pre-analysis of the lightning development in the signal fragment, which could be probably done by some form of identification of lightning phases in the signal. In other words, the creation of a lightning signal model which should be then matched to the real signals from different stations.

\secc Comparison of observed lightning with lightning detection network

The detection of lightning using the VLF has been compared with its detection using the Blitzortung.org network. It's important to note that not every lightning strike detected by the VLF - STP antenna is also detected by the Blitzortung.org network. Figure \ref[figur_blitz2] shows that when the storm was closest to the car according to Blitzortung.org, the STP antenna detected lightning at different times or at distances of more than 120 km. Conversely, as shown in Figure \ref[figur_blitz1], when the storm was located tens of kilometers away from the observation site according to Blitzortung.org, there was perfect conformity with the data from the STP antenna. In both figures, an interval of ±1 second is marked around the vertical lines corresponding to the detection times.

Lightning at 18:24:48 was detected by a high-speed camera, see video \ref[video_1627302288.9546976.mp4]. According to the video, one of the lightning channels occurred almost directly above the measurement car, but the nearest lightning detected by Blitzortung.org was at least 70 km away. The positioning accuracy in the case of the Blitzortung.org network is in the order of kilometers. Blitzortung.org detected discharges at a distance of 70 to 80 km simultaneously. This allows the deduction that this lightning was more than 80 km long or that a synchronous discharge occurred 80 km away.


\medskip
\clabel[figur_blitz2]{Correlation with less than 20 km lightning detections}
\picw=15cm \cinspic ./img/figur_blitz2.png
\caption/f   Correlation with “nearby” (less than 20 km) lightning detection.
\medskip


\medskip
\clabel[figur_blitz1]{Correlation with more than 20 km lightning detections}
\picw=15cm \cinspic ./img/figur_blitz1.png
\caption/f   Correlation with “distant” (more than 40 km) lightning detection.
\medskip

We have developed our own system for radio lightning localization as is shown in section\ref[VLF_LMA], including algorithms for visualizing lightning. However, we were unable to compare our system with those operating in Central Europe. We could not find lightning that could be unequivocally matched with the lightning recorded by our system. We compared our records over the entire duration of storms measured from a single location with the Blitzortung.org network and the LINET network.

Figure \ref[Lighning_trigger_coincidences] shows an example of a severe storm that passed over the measuring vehicle. It indicates the distance of detected lightning strikes from the vehicle's location and all recorded triggers. From the records, we see that only one trigger at 16:43:40 coincides with the lightning indicated near the vehicle. However, we do not have a video recording for this time because the duration of saving data from the camera exceeds 1.5 minutes, so we are unable to capture every lightning. Observers in the vehicle confirmed that there was no lightning near the vehicle at that time. Thus, it was a trigger from a distant lightning. The trigger was confirmed by two measuring vehicles located 2 km apart. The thickness of the lines depicting the trigger is 1 second (±0.5 seconds from the trigger, which is recorded with an accuracy of 10 ms).

\medskip
\clabel[Lighning_trigger_coincidences]{Correlation with more than 20 km lightning detections}
\picw=15cm \cinspic ./img/Lighning_trigger_coincidences.png
\caption/f   Example of a severe storm passing over the measuring vehicle. The figure shows the distance of detected lightning strikes from the vehicle and all recorded triggers. Only one trigger at 16:43:40 matched the vicinity of the vehicle, but no lightning was observed at that time. This trigger was confirmed by two measuring vehicles 2 km apart. The line thickness represents 1 second (±0.5 seconds) from the trigger, recorded with 10 ms accuracy.
\medskip


\sec[conclusions] Conclusions

The research has significantly revised our understanding of lightning phenomena. Initially designed to observe events lasting tens of milliseconds \ref[Initial_assumptions], the apparatus revealed that lightning events typically last at least twice as long, with spatial extents exceeding tens of kilometers. These findings demand a shift in observational methods, requiring recording devices with memory depths greater than one second and observation areas expanded by over 28 times the previously assumed size.

This extended geometry challenges the vertical radiation theory, suggesting that electric fields and RREA likely radiate horizontally or obliquely. Stationary detectors (e.g., GEODOS01 on Poledník) identified a new type of ionizing radiation event lasting tens of milliseconds \mycite[j5], distinct from known TGFs and TGEs.  This may explain why ionizing
radiation could not be detected from below storm clouds using the described measuring vehicles, emphasizing the need for elevated or in situ observations.

Electric field measurements using standard ground-based mills failed to correlate with lightning discharges, underscoring the importance of UAV-based EFM instruments like THUNDERMILL01 and optimized ionizing radiation detectors for in situ observations. Optical and radio data further demonstrated that lightning initiation spans several seconds, far beyond the return stroke phase, which is statistically minor (<10\% of discharges). A focus on understanding the remaining 90\% of lightning events is essential.

Experimental data also indicated that lightning initiation likely occurs far from visible discharge areas. While a potential initiation event was captured on video (appendix \ref[attached_files]), its analysis was limited by missing contextual data, highlighting the need for larger-scale, multi-instrument setups to explore causality between ionizing radiation and lightning.

Finally, the study revealed critical shortcomings in current lightning detection networks. Existing systems oversimplify lightning as clusters of impulses, misinterpret data as ground strikes, and lose essential information. The proposed approach provides a more accurate visual representation of lightning development, marking a significant step forward but requiring further refinement.

\sec Fulfillment of Objectives

\begitems \style n
* {\bf Implementation of Lightning Detection Apparatus}: This research developed and deployed innovative instruments for detecting lightning and ionizing radiation, with significant results published in a Q2 journal \mycite[j1]. A novel pulse-energy detection method, implemented via the time-over-threshold algorithm and described in chapter \ref[triggered_recording], enables reliable lightning-triggered data collection.

Key instruments include novel all-sky high-speed cameras and SDR-based receiver for recording VLF (section \ref[VLF_receiver]) and UHF signals (section \ref[UHF_signal_receiver]), utilizing a novel active direct down-converting QFH antenna, and improved ionizing radiation detectors (GEODOS01 and GEODOS02) designed especially to operate near lightning as is explained in section \ref[ionizing_radiation_detectors]. GEODOS01 deployed on Poledník tower discovered a new type of thunderstorm-related radiation, published in \mycite[j5].

The new Electric Field Mill (THUNDERMILL01), with high time resolution and angular sensitivity, was designed for use on cars and UAVs \ref[Electric_field_mills] published in \mycite[c4]. Together, these instruments form an optimized setup for mobile storm measurements, significantly enhancing data collection and redefining instrumentation standards in storm research \mycite[j1].

* {\bf Enhance understanding of Spatial and Temporal Characteristics of Lightning Discharges}: As discussed in \ref[VLF_LMA] and \ref[conclusions], lightning events cover significantly larger areas and last longer than previously assumed. The VLF signals revealed that recoil leaders, rather than CG return strokes, are the dominant phenomena. These linear objects cannot be reduced to single points, challenging current radio localization systems.

To address this, two new visualization methods were developed. The first, described in \ref[camera_observations], converts high-speed all-sky camera recordings into light curves, enabling efficient comparison of individual lightning records and their correlation with other instruments (e.g., VLF receivers). This method was published in \mycite[j1].

The second method, utilizing VLF signal recordings, maps the branching areas of lightning discharges, as shown in Figure \ref[VLF_lightning_map]. This approach offers a more accurate representation of lightning on maps, potentially replacing the widely used point-based models.

*     {\bf Enhancement of Measurement Techniques with in-situ atmospheric monitoring}: To capture ionizing radiation not only below storm clouds but also from lateral directions, the TF-G2 unmanned autogyro was developed. It can carry a semiconductor ionizing radiation detector and an electric field mill simultaneously, enabling mapping around storm clouds \ref[autogyro_thunderstorm]. However, due to legislative restrictions, most tests were conducted under fair weather conditions or with stratospheric balloons \ref[balloon_flights].

The measuring vehicles were modified to launch the autogyro from the car roof, improving safety, reducing deployment time, and conserving the autogyro's energy for flight. Additionally, the new TF-ATMON system \mycite[j2] enhanced in-situ measurements. Observations revealed that measuring the electric field below and around clouds is essential, as ionizing radiation was detected on high-mountain observatories during thunderstorms passing laterally \ref[figur_efmdatadetail], as we published in \mycite[j5].
\enditems

\sec Future Work

Future research needs to be  focused on in-situ measurements. UAVs, especially unmanned autogyro despite their limited effect in the described work, appear to be ideal for this purpose. It is necessary to measure the electric field in the atmosphere (ideally as a 3D-space vector), which an autogyro, by its rotating rotor - an ideal place for integration of EFM, and high stability in gusty weather could facilitate. That type of in-situ measurement should also include ionizing radiation detection.

The deployment of these approaches in subsequent studies is expected to yield substantial results, possibly confirming the direct detection of RREA and the directional structure of thunderstorm ionizing  radiation sources. That knowledge is essential to interpret the measured ionizing radiation and lightning event data to explain lightning initiation or eventually predict lightning occurrence.

The application of the TF-G2 autogyro in the described study has highlighted the potential and challenges of leveraging UAV technology for atmospheric research. This technology offers a critical area for improvement and optimization in future work. Enhancing the UAV technology will undoubtedly augment higher capability to collect high-fidelity atmospheric data from nearby or within storms, thereby enriching the understanding of thunderstorm dynamics, improving weather prediction models, and contributing to climate studies. Therefore continuous development and field testing of the unmanned autogyro and related technologies are required.
