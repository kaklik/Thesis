Because recording and then processing the large amount of data captured continuously over thunderstorm development is technically difficult (the technical difficulties are detailed in the chapter \ref[instrumentation], specifically in sections \ref[radio_receivers] and \ref[High_speed_camera]), it is needed to have the ability to determine the precise time of the lightning event, with a low rate of false detections. That requirement is enhanced by the fact that data recording is triggered by every false detection, which takes several minutes. This interval creates a “blind” dead time during which the instrumentation is unable to capture another event. To solve that issue, multiple iterations of a mobile measurement system and detectors, which are triggered by electromagnetic emissions produced by lightning, have been used.
The radio-wave detection method is considered to be the most suitable for the intended application, due to its detection range and relative insensitivity to meteorological conditions (hail, snow, rain, and sunlight) at the time of the lightning discharge.

From the signal detection theory, if the signal is known and determined, the signal can be detected by using the signal replication and correlation method or the matched filter method. However, the lightning signal is time-space random and mostly uncertain (see typical lightning signal description in section \ref[lightning_signals]). Therefore there is only the option to use the signal energy to detect the lightning. That approach is examined in the following paragraphs \ref[oscilloscope_trigger] and \ref[manual_trigger].

To detect the signal using its energy, the following basic principles can be applied:
Signal Energy Calculation - The energy $E$ of the signal is given by the integral of the squared amplitude $s(t)$ over the time interval $[t_1, t_2]$:
   $$ E = \int_{t_1}^{t_2} |s(t)|^2 \, dt $$

Signal Energy Detection - To detect lightning, an energy threshold $E_{threshold}$ is determined. If the energy $E$ of the signal exceeds this threshold, a lightning signal is detected:
   $$ \int_{t_1}^{t_2} |s(t)|^2 \, dt > E_{threshold} $$

This formula expresses that a lightning signal is detected when its energy exceeds a predefined threshold value.

Description of Individual Elements:
\begitems
* $s(t)$: Amplitude of the signal at time $t$.
* $t_1$, $t_2$: Start and end time points of the measurement.
* $E_{threshold}$: Energy threshold above which the signal is considered a lightning signal.
\enditems

The proposed system is required to be capable to detection of lightning with precision in the order of several kilometers in position and better than milliseconds in absolute time. The widespread method used for radio lightning detection is based on the detection of the magnetic part of an electromagnetic field generated by lightning. Examples of instruments using this method available for comparison in the Czech Republic include the use of the SLAVIA magnetic loop antenna \cite[Kolmaov2013AnalysisOB] and the stations of the lightning detection receivers of the Blitzortung.org network \cite[Blitzortung].  However, these systems are not designed to be highly portable, nor to be used as a measurement trigger source in offline and stand-alone operations, making it necessary to find a mobile, lightweight, and low-power consumption lightning detector.

Based on the sources in the literature review section \ref[literature_review] the implementation of algorithms for real-time lightning detection in ground-based systems uses simplified version of lightning detection based on the signal energy. In these systems the signal level treshold is set to individual stations by the network server, depending on the analysis of the data stream coming from the station. This approach reduces the complexity of computations and speeds up the detection process.

Instead of integrating the signal energy over time interval, the signal amplitude $s(t)$ is directly compared against a predefined amplitude threshold $A_{threshold}$. If the amplitude exceeds this threshold at any point within the time interval $[t_1, t_2]$, a lightning signal is detected:

$$
\max_{t_1 \le t \le t_2} |s(t)| > A_{threshold}
$$

Description of Individual Elements:
\begitems
* $s(t)$: Amplitude of the signal at time $t$.
* $t_1$, $t_2$: Start and end time points of the measurement.
* $A_{threshold}$: Amplitude threshold above which the signal is considered a lightning signal.
\enditems

That network solution is not suitable for detecting lightning in generic electromagnetic noisy environments and requires station placement in electromagnetically quiet areas. There exists a conference paper \cite[5689931] introducing an adaptive signal detection algorithm, using a short-time energy detection and Constant false alarm rate (CFAR) for real-time VHF intracloud lightning signal detection. This adaptive approach is promising to enhance the performance of lightning detection systems in stand-alone ground-based setups by dynamically adjusting to varying signal characteristics, but unfortunately, the statistics of noise background need to be known. Consequently, that could be hardly used for mobile measurements, and as a result, there is not an existing solution that could be directly used for the planned research. Therefore there are requirement to investigate other possible sources of trigger. The my research begins by using of the standard commercial technology as is described in the appendix \ref[LIGHTNING01A_trigger].
In conclusion, I found out that a lightning trigger based on the currently widely used lightning detection chip AS3935 has limited usability for purposes of lightning recognition and is not able to detect significant amount of discharges. Further search is therefore necessary to find a more reliable way of lightning detection and triggering during thunderstorm. The subsequent research was based on the hypothesis:

Combining these two approaches to mitigate the significant sensitivity (resulting in false positive triggers) of simple amplitude threshold detection in a radio-noisy environment. The combination involves performing partial integration by counting only the samples that consistently exceed the amplitude threshold, and ensuring the number of consecutive samples surpasses a certain minimum length. This approach aims to gather information about the signal energy without the computationally intensive integration over the interval, thus avoiding detection delays.

Identify samples $s(t)$ that exceed the amplitude threshold $A_{threshold}$:
   $$
   |s(t)| > A_{threshold}
   $$

Count the number of consecutive samples $N$ that meet the amplitude threshold condition:
   $$
   N = \sum_{t_1 \le t \le t_2} \chi_{|s(t)| > A_{threshold}}
   $$
   where $\chi_{|s(t)| > A_{threshold}}$ is the indicator function, which equals 1 if $|s(t)| > A_{threshold}$ and 0 otherwise.

Detect lightning if the number of consecutive samples $N$ exceeds a predefined minimum length $N_{min}$:
   $$
   \hbox{Detect lightning if } N > N_{min}
   $$

This combined approach is expected to have benefits of both detector implementations, reducing the likelihood of false positives in noisy environments and improving the reliability of lightning detection without the need for computationally intensive integration to calculate signal energy. This hypothesis was tested by the the following experiments.

\sec[oscilloscope_trigger] Using the oscilloscope to generate a recording trigger

To make the required lightning trigger implementation, a full-featured, fast desktop oscilloscope equipped with deep memory and multiple sophisticate trigger settings based on signal characteristics was utilized. Initially, a simple VLF antenna made of UTP cable loop, originally developed for the Ionozor project, was connected to one of the oscilloscope channels. The antenna design was based on the use of a VLFANT01 module, with the 10 m length, STP cable coiled into four loops (STP antenna). The STP antenna loops were placed horizontally directly at the plywood base mounted on the car roof. See  Figure \ref[loop_antenna] for details.
The oscilloscope was mounted on the back seat of the car by an ISOFIX base and directly powered by a stack of Li-pol batteries totaling 96 V. That solution was selected to completely avoid the possibility of noise generated from the onboard 230 V switching power supply in the testing car.
Subsequently, various trigger settings on the oscilloscope during thunderstorms were tested. The most effective setting turned out to be one that activated the recording (oscilloscope trigger output) based on a combination of threshold level and pulse length. This result is logically justified by the fact that a lightning discharge is a high-energy event involving charge transfer over a significant duration. Therefore confirming the hypothesis stated in the previous section.

\medskip
\clabel[mounted_oscilloscope]{Isofix mount of the triggering oscilloscope in the CAR1}
\picw=15cm \cinspic ./img/oscilloscope.png
\caption/f Oscilloscope used for lightning-triggered recording on board of CAR1.
\medskip

In a basic explanation, the height of the observed electromagnetic pulse corresponds to the rate of change in the magnetic field, and the duration corresponds to the charge amount being transferred. That trigger setup proved itself immune enough that the oscilloscope could be also later powered by 230 V power generated from 12 V car onboard power by switching the power supply, despite the fact the signal recording itself was heavily affected by interference coming from that power supply.
Another advantage of that oscilloscope-based setup was this trigger setting ultimately allowed for the recording of the full length of the lightning discharge signal for up to a few seconds.  However, the price for such a possibility was an extremely long storage-writing time corresponding to many minutes of dead time.
The information gathered from these experiments and subsequently obtained from the recorded signals led to questioning of the assumptions stated in the introductory chapter \ref[Initial_assumptions]. According to the structure of the signal, it was clear that the event was neither temporally nor spatially limited as described by assumptions based on literature and summarised in section \ref[Initial_assumptions]. To confirm this suspicion, however, it was necessary to obtain further supporting data, which led to the need to use an additional range of instruments and to mitigate the significant dead time introduced by oscilloscope recording.

\medskip
\clabel[loop_antenna]{STP based loop antenna prototype}
\picw=15cm \cinspic ./img/VLF_antenna.jpg
\caption/f The one of initial variants of STP loop VLF antenna mounted on the measuring car CAR1 has a resonant frequency of ~100 kHz. Its signal was directly sampled by an oscilloscope placed inside the car.
\medskip

\sec[manual_trigger] Manual trigger method

Although the trigger solution experimentally implemented with an oscilloscope proved effective, it has certain drawbacks. One of the main limitations is the necessity to manually set the two parameters -- pulse level and pulse length for each storm, and sometimes even for each phase of the storm. This characteristic resulted in the omission of some visible lightning events, which the operator in the car was able to see, but the oscilloscope did not yet trigger. Therefore there appeared that a suitable solution would be to add a button for manual trigger operation for all devices, regardless of oscilloscope state.  That was achieved by adding a button to the car's gear lever as is depicted in the photo \ref[Manual_button]. This manual approach not only allows the triggering of instruments before the oscilloscope is set, but it is also a comfortable solution for the driver to trigger while driving in a storm, where the driver often sees lightning in the storm, but configuring the oscilloscope while driving is not possible, as it would also have to dynamically change with the approaching storm.

Experimentally, a solution was also tested where multiple people sat in the vehicle, each visually monitoring their sector for the occurrence of lightning with a button in hand. The result of that experiment showed that the increase in sensitivity with multiple observers was not proportional to the increased manpower required.

However, the described visual-based manual approach has multiple effects on recorded data. One of them is the significantly decreased sensitivity of the human-generated trigger during daytime storms. There are even daytime storms where a person is unable to detect the occurrence of lightning discharges, but the trigger from the oscilloscope is recorded. Another effect, or perhaps more accurately a characteristic, is that such manual trigger is selective and lightning detected by this method contains a higher proportion of return strokes.

\medskip
\clabel[Manual_button]{Manual triggering button mounted on manual gear lever}
\picw=10cm \cinspic ./img/manual_trigger_button.png
\caption/f Manual triggering button mounted on manual gear lever suitable to be operated by the car driver during the thunderstorm.
\medskip

\sec[manual_trigger] Summary of triggered recording methods

The triggered recording method research has concluded that the most suitable solution is an algorithmic approach based on comparing the pulse height to a threshold and measuring its minimum duration before a trigger is executed. Additionally, a manual trigger is required to cover edge cases where the algorithmic trigger is not yet well-calibrated, but there is clear evidence of lightning. The more specific implementation of the algorithmic trigger is discussed in the subsequent chapter \ref[radio_receivers]. The method of lightning detection using the signal energy has been implemented by the pulse energy obtained from VLF antennas. This method is currently the best-known solution. It has been shown that the VLF antenna provides dB/dt, this signal is evaluated by the Time over Threshold (TOT) method, and therefore it is not possible to trigger from very slow discharges. For this reason, the total number of lightning strikes on Earth may be significantly underestimated.
