
To be able to record the different processes occurring during thunderstorms, the three cars were equipped with specialized measuring equipment. The aim is to verify the hypotheses  considering the relation between thunderstorms and ionizing radiation stated in section \ref[ionizing_radiation_detectors]. These cars were able to move to locations with predicted storm activity and thus flexibly react to specific storm developments, and perform ground measurements directly at the lightning site.

To determine the necessary parameters of lightning activity (lightning events’ timestamps, lightning type, and location), the measuring cars, see Fig. \ref[CRREAT_cars] (CAR0 and CAR1), were equipped with high-speed all-sky camera and radio receivers. The cars were used to transport and power the instruments in proximity to thunderstorms, and the car cabin also served as partial protection for the instrument operator.

The equipment of cars is not uniform as it is gradually improved and also the purpose of each car slightly differs.

\medskip
\clabel[CRREAT_cars]{CRREAT measurement cars with instruments mounted on the roof platform}
\picw=15cm \cinspic ./img/CRREAT_cars.jpg
\caption/f The CRREAT measurement cars with instruments mounted on platforms, the overview of differences is shown in diagrams \ref[CRREAT_CAR0_diagram], \ref[CRREAT_CAR1_diagram], and  \ref[CRREAT_CAR2_diagram].
\medskip

\sec[High_speed_camera] High-speed all-sky Cameras

The two cars were equipped with high-speed all-sky cameras. Their purpose is to capture the lightning event evolution in parallel with recording from the other instruments, especially radio receivers.

Chronos 1.4 camera  CR14-1.0-16M \cite[chronos_datasheet] was mounted in a waterproof SolidBox 69200. The box is covered with a plexiglass dome with the manufacturer designation Duradom 200mm depicted in Fig \ref[high_speed_allsky_camera].

\medskip
\clabel[high_speed_allsky_camera]{High-speed all-sky camera}
\picw=15cm \cinspic ./img/high-speed_camera.png
\caption/f The camera has a wide-angle CS camera lens and is equipped with an IR blocking filter mounted in front of the LUX1310 CMOS sensor \urlnote{https://www.luxima.com/product_briefs/PDF/LUX1310_brief_v3.1.pdf}. The IR filter is mandatory, to avoid the adverse effect of sunlight coming into the camera lenses.
\medskip

The video resolution of the camera was set to 928x928 pixels with 1612.33 FPS and a constant shutter in the range of 4.9  to 34 $\mu$s during daytime thunderstorms and a maximum time of 614.6 $\mu$s  for nighttime thunderstorms. The shutter time was set by the instrument operator depending on current weather conditions and the available brightness dynamics of the high-speed camera.  The video-saving length of the camera was set to 2 or 3 seconds. The video save format was initially H.264 resulting in MPEG-4 (.mp4) video, which sacrifices a bit of the quality for better compression. Lately, I found a method to record the 12-bit raw data from the camera without significantly increasing the data recording time.

The camera lenses used were  FE185C057HA-1 \cite[camera_lenses]. The high-speed camera is therefore mainly sensitive to visible light. That spectral baseband was selected to minimize the absorption of light generated by lightning in the atmosphere.

\sec[radio_receivers] Radiofrequency receivers


Radio receiver implementation has several iterations, from the initial overview recordings made with an oscilloscope to a dedicated solution utilizing an antenna array, allowing for a greater number of lightning strikes to be recorded. One of the compromises that had to be addressed in the design of the radio receivers was the choice of the received band. This is limited both by the physical properties of the lightning discharge and by the frequency band allocation table in the EU, as bands with continuous broadcasting are unsuitable for radio observations. At the same time, there are sizing restrictions on the car roof platform, which favors the use of a combination of VLF for detection and UHF for more precision mapping.
The observational band is chosen according to the segment of interest in the lightning discharge. Broadly, the following bands that can be used for observing lightning discharges can be distinguished:

\begitems
\item - VLF - Probably the most utilized from the view of the number of stations globally existing. The band is suitable for observing radiation generated by the main lightning channel. It is not well suited for measuring inter-cloud discharges.
\item - VHF - This band allows the reception of signals generated by initial breakdowns and some subsequent channels. On the contrary, it is not very sensitive to the main lightning channel \cite[https://doi.org/10.1029/JC086iC08p07451, rs13224658, 9226420].
\item - UHF - The band almost exclusively contains information about the initial breakdowns in the early phases of lightning development. The main lightning channel in this band is difficult to observe except at short distances.
\enditems

That fact also favors the combination of VLF and UHF, because the primary interest was set to detection of lightning and then observing the lightning initiation point in coincidence with ionizing radiation.

\secc Recording trigger implementation

As has been already stated, practically all current systems \cite[5173582] detect and locate lightning based on the occurrence of a pulse signal above a specified threshold value. The absolute time of pulse detection is processed, using algorithms based on the Time Difference of Arrival (TDoA) possibly enhanced by a combination with directional findings.  The calculated output is the most probable place of the pulse signal's origin. Only a minority of implementations consider the shape of the pulse or a sequence of pulses for a more detailed analysis of the signal's origin (excluding the correction of pulse dispersion to correct the TDoA value, which is quite common) \urlnote{https://www1.gifu-u.ac.jp/~lrg/falma.html}.

To clarify the phenomena occurring during a lightning discharge, it was necessary to adopt a different approach and implement an SDR receiver with different properties from existing systems. The fundamental differences from other systems include:

\begitems
* The entire segment of the lightning event signal with a bandwidth of 10 MHz is recorded, which is nearly double the bandwidth of the most of currently used solutions.
* Each station uses an antenna array, which can be utilized in the future for advanced signal processing, thereby making it possible to obtain directional resolution.
* Thanks to directional resolution, stations can spatially and temporally separate individual observed phenomena.
* Recording the entire signal segment allows for a more precise reconstruction of the lightning discharge structure.
* Lightning discharge can then be described by vectors in the future instead of the current most common description by a point cloud.
\enditems

For the observation,  the VLF receiver system was used simultaneously for detection of the lightning and triggering other instruments in the cars. The detection of the lightning  was based on pulse width and the signal level; both parameters were set by the operator during the thunderstorm event. The range of these parameters is usually 5 to 20 ~$\mu$s for the pulse width and  10 to 30 mV for the signal level. The values are set directly to the registers of the signal processing FPGA into the receiver. For convenience, the values could be set directly in the web-based UI shown in Fig. \ref[RSMS_control].


\medskip
\clabel[RSMS_control]{Lightnig signal receiver's control software}
\picw=15cm \cinspic ./img/RSMS_control_interface.png
\caption/f Screenshot of the web-based interface to the VLF receiver used to set up the trigger and recording parameters.
\medskip

\secc[VLF_receiver] VLF signal receiver

\medskip
\clabel[VLF_antenna]{VLF antenna array construction}
\picw=15cm \cinspic ./img/VLF_antenna.png
\caption/f A detailed view on aluminum construction that holds orthogonal magnetic loop antennas.  The individual edges are isolated by 3D-printed corners.
\medskip

The VLF signal receiver is based on a magnetic loop antenna and initially on a storage oscilloscope with a control computer for data readout. For one lightning event, that setup could record 800 ms of 8-bit samples with a sampling rate of 250 MS/s taking several minutes of recording time. From that concept which uses the antenna  mounted flat on the plywood as depicted in Fig. \ref[loop_antenna]. In the dedicated receiver, the  antenna array design was introduced, although it is based on the use of the same  VLFANT01 module, with the 10 m length, STP cable coiled into loops. The antenna loops were attached orthogonally to the structure made from 40mm aluminum profiles mounted on the car roof. See  Fig. \ref[VLF_antenna] for details.

The oscilloscope was replaced with very similar hardware as the UHF signal receiver described in section \ref[UHF_signal_receiver]. The main difference is that the VLF antenna is directly coupled to the ADC inputs.

An example of the captured lightning signal can be seen in Fig. \ref[figur_VLFdata].

\medskip
\clabel[figur_VLFdata]{An example of VLF lightning signal}
\picw=15cm \cinspic ./img/VLF_signals.png
\caption/f Data visualization obtained from VLF antennas placed on three measuring cars (each car has a set of three orthogonal antennas). The amplitude and quality of the signal differ significantly. In the first rows, clear interference in the signal on CAR1 caused by switching power supplies is visible. The mutual trigger delay, caused by triggering different signal signatures, is also noticeable, complicating the capture of the entire lightning discharge.
\medskip


The use of this dedicated receiver decreases the data recording interval to about twenty seconds and allows recording of the 1.45-second long signal fragment with 12-bit  ADC resolution.  This increase in parameters was a significant step from the original oscilloscope-based experiments.

\secc[UHF_signal_receiver] UHF signal receiver

The construction of the UHF receiver is designed in a way that enables a phase processing of the signal from the antenna array with the aim of future detailed mapping of the lightning \urlnote{https://github.com/UniversalScientificTechnologies/RSMS01}. However, in the case of this experiment, only the scalar envelope of the radio signal is considered.

The UHF receiver operates approximately in the 370-406~MHz band, exact tuning depends on the local noise situation within the observation area. The receiver has 10~MHz bandwidth. The signal is received by an array of four QFH antennas mounted in a square-like configuration on the roof platform of the measuring car as is shown in the photos in Figure \ref[UHF-QFH_receiver]. The signal from each antenna is directly downconverted by an RF mixer to I and Q analog channels. Each channel is sampled by a 12-bit analog-to-digital converter with 10~MS/s. The block diagram of the receiver is shown in Appendix \ref[SDR_receiver].

\medskip
\clabel[UHF-QFH_receiver]{UHF lightning signal receiver and its components}
\picw=10cm \cinspic ./img/UHF-QFH_Receiver.png
\caption/f The platform mount of a QFH receiver antenna is shown in the top left corner. Detailed pictures of the antenna can be seen at the bottom and the receiver is in the top right.
\medskip


The UHF radio receiver is triggered by an external trigger (coming from the VLF receiver), then records a signal fragment with the configured number of pre-trigger and post-trigger blocks up to 1.45 seconds long in total. The radio signal is simultaneously sampled from all elements in the antenna array. This feature is achieved by using an internal ring buffer that stores the samples from the antenna array before the trigger. Time stamping is implemented by storing an array of sample numbers along with a few last-time marks created by a PPS signal from an external GNSS receiver. This metadata is then recorded in the resulting data record file for each trigger event. That mechanism allows precise restoration of the absolute time of each recorded signal sample.

\medskip
\clabel[CRREAT_CAR0]{Measuring CAR0 with instruments mounted on the roof platform}
\picw=15cm \cinspic ./img/CAR0_instrumentation.jpg
\caption/f Measuring CAR0 with an all-sky high-speed camera and an array of QFH antenna mounted on the roof platform. Other cars were equipped similarly. The differences in instrumentation are depicted in diagrams \ref[CRREAT_CAR0_diagram], \ref[CRREAT_CAR1_diagram], and \ref[CRREAT_CAR2_diagram].
\medskip


The antenna array is mounted on an electrically non-conductive 18 mm thick plywood board attached to the roof of the measuring car by crossbars, located a few centimeters above the car’s metal roof \ref[CRREAT_CAR0]. Each element of the antenna array is based on the quadrifilar helix design. The Quadrifilar Helix Antenna (QFH) consists of two loops. Each loop end is connected in a manner that ensures a 90-degree phase difference between the loops, allowing it to receive signals from all directions without the need for orientation adjustments. This phase relationship is essential for achieving circular polarization, which is required for linear polarization orientation insensitivity (therefore typical application of that antenna is for satellite communication). The loops are arranged in a helical pattern to ensure that the received signal maintains a constant phase across the antenna's bandwidth. This design effectively enhances its performance for applications requiring consistent omnidirectional coverage, which is the situation of lightning detection. That technical feature at the same time allows the direct processing of the signal from the antenna as quadrature I/Q data.

The phase difference of the QFH antenna is extensively exploited in the UHF receiver construction because the actual antenna realization used two loops, which are mechanically joined together on the loop midpoint (to achieve better mechanical rigidity), and loop ends connected to the active analog fronted PCB of the antenna (QFHMIX01) on the other side. QFHMIX PCB is mounted in a metal enclosure made from aluminum alloy. The antenna half-loops pass through the enclosure wall by waterproof cable glands. Each joint of the half-loop and QFHMIX01 PCB is considered to be a 40-50 Ω port. Therefore two single ports with 180° phase shift can be considered as one differential line of approximately 100 Ω. In this case, the QHA antenna has two loops with 100 Ω differential output ports with phase shift 90°.

\medskip
\clabel[QFHMIX_block_schematics]{Block schematics of the UHF active antenna front-end}
\picw=13cm \cinspic ./img/QFHMIX_block_schematics.png
\caption/f Block diagram of active antenna RF-front end. In the left part, the phase-shifted terminals of the QFH antenna are connected, followed by the analog branches for processing I and Q signals.
\medskip

From that perspective, the QHA antenna itself functions the same as an I/Q demodulator. The signal on each port has a 90° signal phase shift in relation to the next port, and QFHMIX01 is actually a UHF analog front-end. Internally, it has two separate signal paths, one for the I signal and one for the Q signal. Each path contains an LNA, mixer, and ADC buffer. The front end also includes a local oscillator amplifier (clock buffer) and power stabilizers.
In total, the PCB of the QFHMIX01 module has two differential 100Ω input ports for the antenna, one differential LVPECL input for the LO signal, and two 100Ω differential outputs for I and Q signals to ADC. It should be noted that LVPECL logic was selected for LO to minimize electromagnetic interference radiated back in antenna loops.  The block schematics of the QFHMIX01 active antenna front-end are depicted in fig.\ref[QFHMIX_block_schematics].

\medskip
\clabel[QFH_antenna_radiation_pattern]{QFH antenna radiation pattern}
\picw=13cm \cinspic ./img/QFH_antenna_radiation_pattern.png
\caption/f Calculated vertical radiation pattern of the single QFH antenna in two perpendicular cross-sections (normalized).
\medskip

The LNA structure contains the blocks concatenated from the antenna input port to the I/Q demodulator input in this way: first BPF, gain block, second BPF, second gain block. First BPF is the filter, whose purpose is to cut out the signals outside of the band of interest, and thereby protect the first gain block from overdriving to a nonlinear regime. The filter passbands are depicted in Figure \ref[input_RF_filter].

\picw=.5\hsize
\clabel[input_RF_filter]{UHF receiver RF filters}
\centerline {a)\hfil\hfil b)}\nobreak\medskip
\centerline {\inspic ./img/input_RF_filter1.png \hfil\hfil \inspic ./img/input_RF_filter2.png }\nobreak
\caption/f Monte Carlo analysis of two RF filters. a) The filter placed immediately after the antenna is a two-stage filter designed to maintain low insertion loss. b) The filter located after the first amplifier is a three-stage filter designed to limit the propagation of out-of-band signals further into the receiver. Both filters are designed using discrete E24 series components, which effectively minimize the possibility of detuning due to vibrations in mobile applications and enable the filters to achieve very compact sizes.
\medskip

The first gain block is based on the MPGA-105 selected for minimal noise figure and high third-order intercept point, specifically, it has NF: 1.8dB, gain: 14.6 dB, and OIP3: 35.8 dB. The second filter is the band pass filter.  The second gain block uses the MGVA-63 selected for acceptable noise figure and high gain while maintaining  a wide dynamic range, it has NF: 3.6dB,  gain: 21.5 dB, and OIP3: 34.3 dB.  This sequence of components gives an RF path with approximately NF: 4.8 dB and gain: 32 dB.

The purpose of the mixer is to convert received signals into the ADC sampling bandwidth. The mixer is based on the AD8343 from Analog Devices. The local oscillator input of the mixer is connected to the clock buffer. The signal output is connected to the ADC buffer.
A clock buffer is needed for the conditioning of local oscillator signals distributed by the antenna cabling from the UHF receiver base station.  The used clock buffer is the Si53322 low-jitter dual output. Each output is used to feed Q and I line mixers separately.
The I/Q signals, which are on the mixer output, need to be connected to the ADC input. To avoid aliasing on ADC it is necessary to pass signals through the LPF to cut off the higher frequencies. For that case, a 5MHz bandwidth LT6604-5 from Analog Devices was chosen. It is also an amplifier for I/Q signals, which allows direct distribution by the UTP cable from the UHF RF front end to the base station inside the car. The output of the amplifier is the 100Ω differential pair, which perfectly matches a twisted pair from the UTP CAT 6 cable.

PCB is produced from the 1.6mm thick FR4. All components, excluding connectors, are SMD mounted from a single side. The RF blocks are carefully separated into different zones to avoid cross-talk. The RF parts are soldered on the bottom side of the PCB. The top side is a shielding GND layer with some signal jumpers and power voltage tracks.

The frontend outputs I/Q signals, local oscillator signal, and power supply are connected to the one IDC381-8-110 punch-down KRONE connector on the top side of the PCB. This connector is the main interface between the RF front end and the receiver base unit mounted in the car.

\medskip
\clabel[QFHMIX01E_PCB]{Active antenna analog front-end PCB}
\picw=10cm \cinspic ./img/QFHMIX01E_top.jpg
\caption/f  Because the design of the QFHMIX01 is optimized for high dynamic range and low noise figure, these tradeoffs result in relatively high power dissipation. The PCB must therefore be equipped with a fan to provide cooling for the analog RF part.
\medskip

It is worth noting that most of the available RF components were chosen to ensure that the design can operate across a wide range of frequencies without significant modifications. Specifically, to retune to a substantially different frequency, it is necessary to replace the conductive structure of the antenna loops and retune the input RF filters in the QFHMIX01. This approach allows for the easy assembly of a system that can operate anywhere in the range of 40 MHz to 1 GHz. Additionally, these changes can be implemented on existing manufactured units, allowing the same equipment to be used in subsequent experiments without substantial investments.


\sec[ionizing_radiation_detectors] EMI-resistant Ionizing radiation detectors

During the research, scintillation, and silicon semiconductor ionizing radiation detectors were primarily used. The scintillation detectors comprised a combination of NaI(Tl) scintillator and Silicon Photomultiplier (SiPM) sensors to eliminate the potential influence of magnetic effects from lightning discharges on the detectors. The detector's design was derived from the existing AIRDOS detectors in the case of semiconductor detectors and AIRDOS-C \cite[Velychko_Kákona_Ambrožová_Ploc_2021]  for scintillation detectors. These detectors were an almost instant solution to cover a broad deposited energy range from 200~keV to 40~MeV with a time resolution down to 100 µs for high-energy events (>1MeV), ensuring detailed temporal and spectral analysis of ionizing particles. For lower energies, only 15~s integration of events is provided.

Newly derived scintillation detectors GEODOS01 and GEODOS02 had scintillation crystals with a diameter of approximately 18~mm and a length of 30~mm. In addition to these newly developed detectors, the commercial RT-56 detectors were also operated on vehicles for a short period. However, due to their internal and unknown autocalibration, which is unsuitable for mobile measurements, they were subsequently moved to static observation sites. The identical construction of GEODOS01 detectors was then also used at static observatories, similar to the RT-56 detectors. All these described types of detectors at static locations detected ionizing radiation related to thunderstorm activity. Specifically, the observation sites were Poledník and the Lomnický Štít observatory.
Unfortunately, the mobile deployment of these detectors in the measuring cars does not provide any significant detections of ionizing radiation, therefore the system was widened by the use of  SPACEDOS with another silicon PIN diode to increase sensitivity.  The G-M tube described in subsequent balloon experiments was also used for a short time, but none of these sensors were able to indisputably detect ionizing radiation associated with storm activity during mobile measurements. The causes of this situation will be discussed in chapter \ref[results].
Towards the end of the research, this approach was reassessed, leading to the design of another ionizing radiation detector. In this latest design, a larger NaI(Tl) crystal was optically coupled with a USTSIPM01 module, whose electrical signal output is connected via a balun transformer \urlnote{https://www.mlab.cz/module/SATABAL01/} to a spare ADC channel of the previously described VLF receiver. The result is a device capable of recording individual ionizing radiation interactions in the NaI(Tl) crystal, timely coupled to the lightning. The maximum recording length is identical to the length of the recorded VLF signal fragment.

The screenshot in Figure \ref[scintillator_pulse] shows an example pulse resulting from cosmic radiation outside of storm activity. Unfortunately, this approach was implemented towards the end of the measurement season, during which cars were available. Consequently, there is very little experimental data, therefore this potential improvement of sensitivity was not adequately examined.

\medskip
\clabel[scintillator_pulse]{Example of the pulse from scintillator detector}
\picw=15cm \cinspic ./img/scintillator_pulse.png
\caption/f  Example of an analog pulse from a silicon photomultiplier (USTSIPM01 module) that is digitized by a free channel on the ADC of a VLF receiver. The pulse is probably caused by a muon from cosmic rays.
\medskip

\sec[Electric_field_mills] High Time Resolution Electric Field Sensors

Electric field sensors were planned to be used as devices capable of indicating the potential for lightning discharges in clouds. For this purpose, Boltek detectors were initially acquired and Boltek detectors were used for ground-based measurements. Despite their widespread use, these detectors presented significant limitations for mobile measurements due to their bulkiness and less optimal design for mobile deployment. Seeking a more adaptable solution, the more compact Kleinwächter EFM 115 was transitioned to. This Electric Field Mill (EFM) offered an improved form factor for mobile applications, directly connecting its analog signal output to a data logger equipped with a GPS for accurate timestamping \urlnote{https://github.com/mlab-modules/FIELDMILL01}, with a logging time resolution of 110 ms.

However, challenges persisted with the Kleinwächter EFM 115, as its mechanical and weather durability did not meet the demands of mobile fieldwork, and time resolution was not adequate to observation of lightning events as I published in \mycite[j1]. This situation resulted in the necessity for an alternative solution, prompting the development of a new design for electric field mills. The new design also required increased versatility because these custom mills were engineered to integrate into stationary setups, mobile measurements vehicles, and even atmospheric monitoring via Unmanned Aerial Vehicles (UAVs).

For the construction of the new electric field mill (THUNDERMILL01), the initial design was first tested in a laboratory and then installed on the roof of a measuring car, as shown in Figure \ref[THUNDERMILL01_on_car]. At that time, the mill was housed in a tin-plated can turned upside down to protect it from external weather influences. This solution proved ineffective, as the metal can rusted through in approximately two months. Subsequent units intended for stationary placement at the Lomnický Štít observatory were therefore made with a stainless steel casing, as shown in Figure \ref[LS_THUNDERMILL]. The measured values are compared with the Boltek sensor, as illustrated in the graph in Figure \ref[THUNDERMILL_LS_boltek]. The results from that research are published in \mycite[c4].

\medskip
\clabel[THUNDERMILL01_on_car]{EFM mounted on the CRREAT car's roof}
\picw=15cm \cinspic ./img/THUNDERMILL01_on_car.png
\caption/f  THUNDERMILL01 electric field mill housed in a tin-plated can, mounted on the roof of a car to test its performance and weather resistance in an outdoor setting.
\medskip

\medskip
\clabel[LS_THUNDERMILL]{THUNDERMILL01 mounted on Lomnický Štít observatory}
\picw=15cm \cinspic ./img/THUNDERMILL_LS.png
\caption/f  Installation of THUNDERMILL01 at Lomnicky Štít Observatory, the electric field mill is installed here in both down and up orientations. This experiment aimed to measure the charge of the hydrometeors to explain how the charge in the cloud is generated.
\medskip

After testing on the car roof, there was a requirement to deploy the same electric field mill to the rotor head of the TF-G2 autogyro to measure the distribution of the electric field directly in the atmosphere. This placement can be seen in Figure \ref[E_mill_rotor] in section \ref[autogyro_thunderstorm], where the UAV instrumentation is described in more detail.

\midinsert
\clabel[THUNDERMILL_LS_boltek]{Comparison of Boltek and THUNDERMILL01}
\picw=15cm \cinspic ./img/THUNDERMILL_LS_boltek.png
\caption/f  The graph shows a comparison of the UP and DOWN configurations of THUNDERMILL01 at the Lomnický Štít observatory. Additionally, it includes a comparison with the particle flux of ionizing radiation detected by the SEVAN detector, also located at Lomnický Štít. The electric field measurements are displayed for Boltek and THUNDERMILL instruments for comparison.
\endinsert

\sec[meteo_instruments] Meteorological instruments

To accurately assess the meteorological situation during thunderstorms, the measurement vehicles are equipped also with a set of standard meteorological instruments. These allow for the real-time monitoring of environmental parameters used for improved understanding of condition evolution. Below is an expanded overview of additional instruments, incorporating details from the provided references.

\secc Precipitation Distinguishing Disdrometer

The implementation of the disdrometer was necessitated by the need for high-time resolution measurement of precipitation patterns and their evolution. Precipitation is a key meteorological parameter  for understanding the correlations between ionizing radiation and storm activity, because hydrometeors play a significant role in washing radon decay products out of the atmosphere, consequently creating a temporary increase in radiation background levels on the ground \ref[Radon_precipitation]. High-rate precipitation data are necessary for distinguishing these events from occurrences of TGE.

\medskip
\clabel[Radon_precipitation]{Radon progeny washout}
\picw=15cm \cinspic ./img/Precipitation_radon.png
\caption/f  The graph demonstrates the impact of rainfall on ionizing radiation flux on the ground. The top panel shows the detections from the LIGHTNING01 module (described in section \ref[LIGHTNING01A_trigger]), and distance to lightning strikes (Blitzortung.org), while the subsequent panels display counts per second for different scintillation detectors (AIRDOS-N inside car, RT-56 on the car roof). The final panel indicates precipitation intensity (5T precipitation). Notably, there is a clear increase in radiation counts following periods of rainfall, highlighting the washout effect of radon progeny from the atmosphere.
\medskip

Traditional, widely used methods of precipitation measurement have their limitations. For instance, the commonly used tipping bucket rain gauge only provides data resolution dependent on precipitation intensity and has a relatively narrow effective range where the collection bucket can tip and accurately measure rainfall.

\medskip
\clabel[tipping_bucket_limitation]{Example of tipping bucket rain-gauge data}
\picw=15cm \cinspic ./img/tipping_bucket_limitation.png
\caption/f  This graph illustrates the limitations of using a tipping bucket rain gauge. While there is a slight increase in radiation detected (around 11:00), the lack of detailed information on precipitation type and better temporal resolution makes it difficult to attribute this increase to a specific phenomenon.
\medskip


Furthermore, the tipping bucket rain gauge does not differentiate between different types of precipitation (rain, snow, hail, graupels, etc.) as could be seen in Figure \ref[tipping_bucket_limitation]. The different hydrometeors types have an impact on the efficiency of radionuclide washout and also the electrification of storm clouds. The tipping bucket rain gauge has a significant advantage in precise volume measurement of the precipitation, but it is a variable that is not useful for the proposed thunderstorm observation because rain intensity and hydrometeor type are more important. In contrast, the disdrometer---another type of instrument---is capable of identifying precipitation types but generally lacks precision in measuring volume. In the case of mobile measurement using cars, there is also a requirement to avoid optical principles, which most commercial disdrometers are based on, which are unsuitable for vehicle roof mounting due to their sensitivity to movement and vibration.

In response to these challenges, the DISDROMETER01\urlnote{https://github.com/UniversalScientificTechnologies/DISDROMETER01}, was specifically designed for installation on a flat vehicle moving platform, enabling high-precision, real-time data collection during storm events. This instrument leverages SDR receiver technology previously utilized in the Bolidozor meteor detection network. It employs a piezoelectric element (see Figure \ref[DISDROMETER01]) capable of  sensing of raindrop size distribution and intensity. The sensing element responds to different hydrometeors with different waveforms as is shown in Fig. \ref[DISDROMETER01_hyrometeors].

\medskip
\clabel[DISDROMETER01]{DISDROMETER01 internals}
\picw=15cm \cinspic ./img/DISDROMETER01.png
\caption/f  Photograph of a DISDROMETER01 mounted on the roof of a car, featuring an open box that contains the piezo element, the primary sensing component of the device.
\medskip

Technically the disdrometer captures sound waves generated by the impact of precipitation particles on the plastic enclosure \ref[DISDROMETER01], providing insights into the characteristics of rainfall during time.


\medskip
\clabel[DISDROMETER01_hyrometeors]{Hail and raindrop captured by DISDROMETER01}
\picw=10cm \cinspic ./img/DISDROMETER01_hydrometeors.png
\caption/f  The graphs show the difference in signal responses between various types of precipitation, specifically illustrating the difference between rain and hail. The top image is a detailed view of the impact of a single hailstone and below a single raindrop.
\medskip


\secc Storm-resistant Anemometer

Given the necessity of knowing both the direction and speed of the wind during storm measurements, especially for launching unmanned aerial vehicles from the car rooftops and for overall safety in measurement, it was essential to equip the measurement vehicles with an anemometer capable of assessing these parameters in storm conditions.
The development of the alternative anemometer was initiated in response to the failure of a standard cup anemometer during its first deployment on a highway drive. This unexpected failure highlighted the need for a device capable of enduring high-speed conditions (at least 130 km/h) in adverse weather, including wind and hail, without significant maintenance. Although the issue might have potentially been addressed by acquiring a high-quality professional anemometer, a more application-suitable design, with less aerodynamic drag was sought, given the intended use case. This pursuit coincided with efforts to develop a replacement for the Pitot tube for UAV applications (described in detail in section \ref[autogyro_thunderstorm]). Logical progression led to the use of similar technology for the anemometer mounted on a vehicle.

\medskip
\clabel[WINDGAUGE03]{ WINDGAUGE03 anemometer mounted on the car roof}
\picw=15cm \cinspic ./img/WINDGAUGE03.png
\caption/f  A photograph showing the installation of the WINDGAUGE03 anemometer on the car roof alongside a windsock.
\medskip

The new anemometer's design benefits from the inclusion of a circuit board that not only accommodates a differential pressure sensor but also a magnetometer. This enables the measurement of absolute wind direction without the need for directional calibration against the actual vehicle's orientation, an advantage over conventional weather station setups that typically require fixed positioning relative to the north. Although the later integration of the RTK GNSS receiver in measurement car infrastructure (depicted in figure \ref[RTK_moving_baseline]) partially reduces this requirement by eliminating the need for direction correction, the WINDGAUGE03 anemometer has been used further due to its resistance to adverse weather conditions.


\medskip
\clabel[WINDGAUGE03_measurement]{Example of data measured by the designed anemometer}
\picw=15cm \cinspic ./img/WINDGAUGE03_measurement.png
\caption/f The graph illustrates data measured by the new anemometer WINDGAUGE03 at the edge of a storm. The top panel shows lightning strike data from Blitzortung.org, the middle panel presents wind direction measurements, and the bottom panel displays wind speed data from the anemometer.
\medskip


\secc Thermometer and Barometer

Temperature data is used in multiple ways, one of the most common is for eliminating possible temperature-dependent variations that might influence the sensors mounted on the car platform.  For that purpose the MLAB ALTIMET01A module was used, which combines temperature and pressure sensing capabilities, providing an overview of the atmospheric conditions.

\medskip
\clabel[temperature_pressure]{Example graph of temperature and pressure evolution during thunderstorm}
\picw=15cm \cinspic ./img/temperature_pressure.png
\caption/f  The graphs illustrate the changes in temperature and pressure during the passage of a thunderstorm. As evident, both variables fluctuate throughout the atmospheric event, often in a significant manner.
\medskip
