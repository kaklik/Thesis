\sec[problem_definition] Problem Statement

The activity of storms is associated with many phenomena whose nature is not yet fully understood or clarified. This includes the process of thundercloud electrification and the subsequent electric discharges, which are the most prominent manifestation of \ii thunderstorm thunderstorms. As a result, weather forecasting and nowcasting of storm activity and related dangers are often very unreliable.

As one of many related phenomena, storm activity is also associated with the generation of \ii ionizing/radiation ionizing radiation \cite[natural_high-energy, Siingh_cosmic_rays, Xray_enhancements, McCarthy1985-tc, Xray_electric_field,gamma-ray_dose_increase,Ground-based_observations_thunderstorm]. It is assumed that the source of this radiation is bremsstrahlung generated by electrons accelerated by an electric field in the thunderclouds \cite[gamma-ray_dose_increase,Ground-based_observations_thunderstorm,Photonuclear_reactions_triggered,Termination_electron_acceleration,Ebro_delta_region]. These electrons that are accelerated to relativistic velocities are called relativistic runaway electron avalanches (RREAs), which are then interacting with the atmosphere \cite[Dwyer_2003, Gurevich_Milikh_Roussel-Dupre_1992]. This causes a phenomenon that is to be often called \ii terrestrial/gamma-ray/flash/(TGF) terrestrial gamma-ray flash (TGF) \cite[Fishman_Bhat_Mallozzi_Horack_Koshut_Kouveliotou_Pendleton_Meegan_Wilson_Paciesas_et al._1994] or another phenomenon like \ii thunderstorm/ground/enhancement/(TGE) thunderstorm ground enhancement (TGE) \cite[Chilingarian_2013, Torii_Takeishi_Hosono_2002]. For both, the source of the radiation is thought to be somewhat associated with the RREA; the main difference is the time span in which they occur \cite[Dwyer_2003, Gurevich_Milikh_Roussel-Dupre_1992], although it has not been experimentally proven. Furthermore, recent experimental results show that there is evidence of other ionizing radiation manifestations generated by the thunderstorms. For example, there has been experimental evidence of the interaction of high-energy photons with the atmosphere causing nuclear reactions \cite[Enoto_Wada_Furuta_Nakazawa_Yuasa_Okuda_Makishima_Sato_Sato_Nakano_et al._2017].
The ionizing radiation that is thought to be associated with storm activity was measured using satellites in orbit around the Earth (e.g. \cite[Østgaard_Neubert_Reglero_Ullaland_Yang_Genov_Marisaldi_Mezentsev_Kochkin_Lehtinen_et al._2019] ), aircrafts flying inside or in the vicinity of storm clouds \cite[Kochkin_van Deursen_Marisaldi_Ursi_de Boer_Bardet_Allasia_Boissin_Flourens_Østgaard_2017, McCarthy_Parks_1985, Parks_Mauk_Spiger_Chin_1981] or high mountain observatories e.g. \cite[Chilingarian_Hovsepyan_Hovhannisyan_2011, Chum_Langer_Baše_Kollárik_Strhárský_Diendorfer_Rusz_2020, Tsuchiya_Enoto_Torii_Nakazawa_Yuasa_Torii_Fukuyama_Yamaguchi_Kato_Okano_et al._2009]. Currently, there are only a few measurements that could confirm the existence of ionizing radiation at lower altitudes, except for special storms that occur during the winter season in Japan, where storm clouds emerge low above the ground \cite[Michimoto_2007].

One of the most interesting measurements performed until now is the combination of radio signal and ionizing radiation. That experiment includes the mapping of a radio signal emitted by lightning \cite[Rison_Thomas_Krehbiel_Hamlin_Harlin_1999, Wu_Wang_Takagi_2018].
There also exist TGF observations with a simultaneous \ii lightning/mapping lightning mapping using radio signals \cite[Abbasi_Abu-Zayyad_Allen_Barcikowski_Belz_Bergman_Blake_Byrne_Cady_Cheon_et al._2018]. Despite these very detailed observations, the exact location of the source of the ionizing radiation emergence as a result of electric fields within the storm cloud remains unknown \cite[2019JD031940]. Moreover, the TGFs were only rarely successfully measured at the ground level \cite[2021JD036130, 2012JA017810, 123061103].

It also should be noted that lightning phenomena manifest not only in the familiar forms that could be observed from ground level but extend into near-Earth space, presenting phenomena such as \iid sprites, \iid elves, \ii gigantic/jets gigantic jets, and also TGFs. All of these are powered by the intense electromagnetic and quasi-electrostatic fields related to lightning discharges. However, the specific properties of lightning discharges that lead to these high-altitude phenomena remain a subject of ongoing research, with studies leveraging both ground- and satellite-based observations to map global occurrence rates \cite[7180031].
At the ground level the lightning discharges that could be observed, are classified into negative, positive, and bipolar. The taxonomy of lightning includes a range of discharges: cloud flashes (intracloud, intercloud, and cloud-to-air) and cloud-to-ground (CG) discharges, the latter accounting for about 25\% of global lightning activity. CGs predominantly consist of negative downward lightning, where a negative charge is transported from the cloud to the ground \cite[Rakov_2016].
Observations in tropical regions have introduced further classifications, including \ii intra-cloud/discharge intra-cloud discharges, cloud-to-cloud, cloud-to-air, and express the polarity of lightning by ground-to-cloud, and cloud-to-ground discharges, noting the significant damage caused especially by cloud-to-ground and ground-to-cloud flashes \cite[mehranzamir2014].
Research into specific phenomena like \ii ball/lightning ball lightning, sprite lightning reveals the versatility and is out of scope of this thesis, opening possibilities for further exploration of these rare and unique events \cite[doi:https://doi.org/10.1002/9780470030875.ch6].
While the general mechanics of lightning—its initiation and propagation—have been linked to specific atmospheric conditions such as the presence of graupel, ice, and hail, highlighting the relationship between lightning types and the microphysical characteristics of the convective regions, many aspects, including the differential occurrence rates and damaging potential of positive versus negative discharges, still invite further investigation \cite[ribaud:hal-01386421].


Intracloud lightning discharges are known for their characteristic radio pulses, which consist of a uniform burst pattern. These bursts are described as a distinctive waveform characterized by a fast, large amplitude pulse followed by a smaller, slowly varying overshoot. The full width at half maximum of these pulses measures 0.75 $\mu$s, with inter-pulse intervals of 5 $\mu$s \cite[https://doi.org/10.1029/JC080i027p03801].
The \ii compact/intracloud/discharge compact intracloud lightning discharge (CID), a particular type of intracloud lightning, is described as a bouncing-wave phenomenon. This involves multiple reflections occurring at both ends of the radiating channel, contributing to its fine structure and accompanying high-frequency (HF) and very high-frequency (VHF) radiation bursts \cite[5184875].
An interesting characteristic of radio frequency emissions during thunderstorms is their nature compared to weaker emissions. The strongest pulses typically occur in isolation or at the beginning of leader progression. These pulses are sometimes associated with rapid electric charge relaxation and are not necessarily accompanied by visible light emissions. In instances where these pulses initiate, they are followed by an upward-progressing leader \cite[https://doi.org/10.1029/2003JD003936].
The study of lightning-induced \ii radio/pulses radio pulses has been also expanded by observations from the satellites, which identify additional characteristics in these emissions. Some exhibit steep roll-offs of power within certain frequency ranges, while others demonstrate flat-spectrum behavior. This distinction indicates the somewhat varied nature of lightning’s electromagnetic emissions \cite[https://doi.org/10.1029/1998RS900043].
For \ii cloud-to-ground/flash cloud-to-ground (CG) flashes, the structure typically includes a sudden start with a stepped leader, in contrast to \ii cloud-to-cloud/flash cloud-to-cloud (CC) flashes that initially showcase a slower train of noise pulses. These RF radiation patterns from lightning display a distinct structure based on the type of lightning flash, differentiating between the abruptness of CG flashes and the gradual initialization of CC flashes \cite[LeVine1978TheTS].

\sec[lightning_signals] Key radio signatures of lightning events

\secc K-changes

\iid K-changes, or K-complexes, refer to a specific pattern of rapid waveform of electric field change observed. The term K-change is used today to denote relatively small step-like static field changes that occur in between and after return strokes, and also during intra-cloud flashes \cite[Nanayakkara]. A \ii return/stroke return stroke is the high-current electric discharge in lightning, occurring when the downward leader connects with an upward streamer, rapidly neutralizing the charge between ground and cloud  and creating a bright, intense flash. This process involves currents typically reaching tens of thousands of amperes, propagating upwards at one-third the speed of light \cite[rakov_uman_2003].

\secc Narrow Bipolar Events (NBEs)

NBEs are distinct, intense radio pulses with a very short duration, typically a few microseconds, and are considered the most powerful natural VHF (Very High Frequency) sources in the Earth's atmosphere. They exhibit a remarkably narrow bipolar pulse shape and are believed to result from a rapid discharge process within thunderclouds, possibly associated with the initiation stages of lightning \cite[Wu2012DischargeHO].

\secc Sferics

\iid Sferics is short for atmospherics, the term used for radio waves emitted by lightning discharges. These signals, spanning a broad range of frequencies but most commonly observed in VLF and LF (Low Frequency) bands, represent the electromagnetic signature of distant lightning's return stroke. Sferics carry distinctive timing information, making them valuable for long-range lightning detection and location systems \cite[Koochak].

\sec[literature_review] Literature Review and state-of-the-art

Atmospheric electromagnetic phenomena have been identified as mandatory physical properties influencing atmospheric radiation processes. These phenomena, including lightning, sprites, and other events, are also a significant part of  the formation of cloud structures and significantly impact the movement and characteristics of atmospherically relevant subsystems \cite[npg-20-293-2013].
The evolution of radio transmission has been one of the earliest methods of regular observation of atmospheric electricity (e.g. sferics). It has rapidly revealed the existence of atmospheric charge structures, such as the ionosphere and its variations, which form the atmospheric waveguide allowing the propagation of radio waves produced by lightning over long distances. Research has also delved into the dynamics of geomagnetic pulsation regimes and their interaction with the Earth's magnetosphere, reflecting the interactions resulting from powerful solar flares \cite[Parkhomov2021GeomagneticPB].
In parallelly ongoing meteorological research, the integration of advanced technologies from aviation starting from balloons led to the discovery of cosmic radiation \cite[Hess:1912srp, REGENER1935]. The balloon observations were combined with radio transmission to be automated in the form of radiosondes \cite[8561fc97-ffa1-39dd-ae09-da46857c3d48, https://doi.org/10.1029/2006JD008187]. That technology has been improved and used for decades, until the most recent stage of  evolution in  Unmanned Aerial Vehicles (UAVs) \cite[CHANG2020126867].
A significant contribution of  radio engineering in meteorology comes from the invention of the radars, which are used for remote sensing of atmospheric structures.  Remote radio sensing and in-situ measuring methods are also combined to obtain multiple perspective views on electric phenomena \cite[Fabry_2015, https://doi.org/10.1029/2018JA025622, bb278c67c86b4412b264b47fd73da169].
The increasing requirements on lightning safety during the Space Race significantly enhanced the technology of ground-based equipment such as electric field mills and radio-based lightning detection and mapping systems \cite[LeVine1978TheTS, Amman1970, 5173582]. The state of art of this radio-based technology utilizes large area EFM arrays for observation in the form of non-intrusive methods to determine atmospheric electric fields and also mapping of lightning evolution and structure. Notable examples of these systems are New Mexico Tech's Lightning Mapping Array (LMA) \cite[https://doi.org/10.1002/2016JD025159],  FALMA \cite[https://doi.org/10.1002/2018GL077628], community-based network Blitzortung.org \cite[Blitzortung.org] and also LOFAR radio telescope  \cite[https://doi.org/10.1002/2017JD028132].
Based on the mentioned instruments, which were originally developed directly for scientific purposes or safety engineering, many commercial products and services for civil lightning awareness have also emerged. As examples, there could be mentioned the Vaisala lightning sensors \cite[vaisala],  WWLLN \cite[https://doi.org/10.1029/2021RS007293] and LINET \cite[Betz2009].



\secc[Initial_assumptions] Summary of Initial Assumptions

The following scientific assumptions provide the foundation for the following research. While they represent the initial understanding in the field, this dissertation will critically examine and test these assumptions to enhance or refine the knowledge of thunderstorm phenomena:

\begitems
* Research in phenomena in thunderstorms, such as ionizing radiation production has often focused on the peaks of electric fields associated with the initiation of lightning [Kolmašová, et. al., 2015]
* Lightning discharges are generally considered to be short-duration events  typically ranging from tens of microseconds [Kolmašová, et. al., 2015], to up to 350 milliseconds [Rakov \& Uman, 2003; López et al., 2017]
* The spatial extent of lightning discharges is believed to be limited to approximately 15 kilometers from the initiation point [Thottappillil et al., 1992; López et al., 2017]
* It is commonly accepted that ionizing radiation is generated before or during lightning events, primarily in a vertical direction parallel to the electric field intensity vector  [Enoto et.al., 2017]
* Much of the existing research on lightning discharges focuses on the return-stroke phase, a key phase of cloud-to-ground discharges [Uman, 1969; Berger et al., 1975; Dwyer et.al. 2012]
\enditems

\sec Research Methodology and Planned Approaches

Based on the theoretical framework and assumptions outlined in section \ref[literature_review], I proposed the research methodology. This methodology includes the use of measuring vehicles and unmanned aerial vehicles (UAVs) to improve the detection of radiation phenomena associated with thunderstorms. The primary objective is to position detection devices as close to or beneath the storm core as possible, thereby increasing the likelihood of capturing relevant radiation events. It is based on the following key components:

\begitems
* {\bf Utilization of Measuring Vehicles:}
I use measuring vehicles equipped with specialized instruments for detecting radio signals and ionizing radiation. These vehicles also serve as mobile platforms, providing the necessary ground support for the UAVs. They are at the same time equipped with a range of auxiliary devices to capture the interrelationships between the measured data. For example, scintillation detectors for ionizing radiation must account for the influence of temperature changes on sensitivity, and disdrometers are used to measure the intensity and type of precipitation that can wash radon products from the atmosphere, causing increased radiation levels, which could easily inappropriately interpreted as TGE.
* {\bf Deployment of UAVs for Radiation and Electric Field Detection:}
Multiple UAVs are equipped with electric field mills (EFMs) and detectors for ionizing radiation and are tasked with locating and identifying sources of ionizing radiation in the atmosphere. This method leverages the strong presumption that the intensity of generated ionizing radiation diminishes with increasing distance from its source. The UAVs provide data on ionizing radiation that is difficult to obtain from ground-based measurements alone. Therefore UAVs are supposed to observe thunderstorms cooperatively with measuring cars.
\enditems

The flowchart depicted in Fig. \ref[workflow] illustrates a high-level workflow of the research. It begins with the setup of measurement cars equipped with radio lightning detection and recording capabilities including an all-sky high-speed camera on the two cars. Experimental measurements are then conducted under thunderstorms, with recorded lightning events analyzed for the presence of ionizing radiation. These attempts  lead to insights such as the flat, large object nature of lightning, the possibility of lightning initiation process over large areas, and the observation that ionizing radiation does not radiate strictly downward.
Parallel to this, the UAV autogyro and other instruments are designed to aid in these measurements. Observations are then supposed to be made from higher altitudes or side angles to enhance data collection. Then the data from both the measurement cars and UAVs  needs to be combined.

\medskip
\label[workflow]
\picw=15cm \cinspic ./img/workflow.pdf
\caption/f Simplified workflow for detecting and analyzing lightning radio signals and ionizing radiation using measurement cars and UAVs.
\medskip

To address the general unavailability or limitations of commercially available instruments, as well as issues with data quality due to closed design and lack of transparency in internal data filtering processes, I developed several bespoke instruments. These custom-made tools were essential to ensure reliable measurements. Notably, the following instruments were developed to meet the specific needs of this research:

\begitems
* {\bf Lightning Radio Signal Receivers:}
These instruments were used to implement a method for triggering data recording by lightning signals, ensuring high sensitivity to natural lightning while minimizing false positives from man-made signals. This innovative solution is detailed in chapter \ref[triggered_recording].
* {\bf Unmanned Autogyro for atmospheric measurements:}
To address the lack of available unmanned systems capable of measuring ionizing radiation or electric fields within storm clouds, I developed an unmanned autogyro designed for atmospheric measurements under various weather conditions. Unlike traditional measurement balloons, which are unsafe and yield sporadic results due to limited control, the autogyro offers precise control and stability even in turbulent conditions.
* {\bf Measurement Support Systems:}
Design and integration of an instrument toolset that allows simultaneous detection and recording of lightning event signals, electric field distribution, ionizing radiation, camera recordings, and meteorological data. The detailed design of these instruments is presented in chapter \ref[instrumentation]. Measuring vehicles are also equipped with systems for wind speed and direction measurement, and sensor orientation recording, necessary for UAV takeoff from the car platform, landing, flight control, and lightning localization.
\enditems

The contributions of the designed instruments to the understanding of thunderstorm-related phenomena are demonstrated through various experiments. These experiments are described in detail in chapters \ref[experiments] and \ref[results]. The measuring vehicles and UAVs' ability to relocate flexibly and conduct ground measurements at storm sites significantly enhances the future capability to capture and analyze storm-related ionizing radiation events.


\sec Outline of the thesis

The initial work of the thesis begins with chapter \ref[triggered_recording], which describes the construction and test of a device engineered to automatically produce trigger signals based on lightning events. The chapter has been one of the initial work packages because the described methods enable the recording of instruments synchronized with the lightning. That is the essential feature needed for every additional step of the following research.

This is attributed to the fact that within the scope of storm evolution, lightning can be seen as occurring in relatively brief and sparse periods compared to the more extended spans of storm activity without lighting (and supposedly also without ionizing radiation), as discussed in sections \ref[literature_review] and \ref[problem_definition]. Simultaneously, the flow of individual ionizing radiation particles or photons during TGF or TGE spans a broad spectrum of possible energy levels. That does not permit the setting of an exact recording threshold at the individual instrument, given the sensitivity capabilities of available equipment. Therefore, ionizing radiation detectors, and the majority of other instruments, like radio receivers and high-speed cameras, are required to capture event data throughout its actual occurrence, and this data fragment should subsequently be analyzed to examine the underlying phenomena.

Therefore these instruments are at the same time examples of scientific devices capable of generating an enormous amount of data. Such volumes of data are (with the actual technology level) difficult to store within the intervals between events, e.g. lightning. That is the main rationale behind the need to create a reliable method of triggering the recording of all instruments depending on lightning activity to store event-based data fragments.

Chapter \ref[instrumentation] explored the exact design of the innovative tools and devices engineered especially to improve the study of lightning in thunderstorms and the related atmospheric phenomena. The design and development of these instruments were initially guided by literature listed in chapter \ref[literature_review], which suggested typical lightning properties that early experiments soon called into question. This suspicion led to multiple iterations in the construction of instrumentation, as the original assumption of lightning's brevity—thought to correlate the event data with the vertical development of lightning channels over tens of milliseconds—proved grossly underestimated, as the actual ranges of these events are several orders of magnitude greater. This situation also changes the requirement on examined electric field gradients, because the vertical gradient detection needs to be exchanged by a horizontal or more ideally the 3D space distribution.  The chapter discusses the rationale behind the instrument development, its design principles, and how these advancements improve upon existing technologies. It also details the experimental setup and integration of these tools into broader research methodology, emphasizing their future potential to significantly advance the understanding of atmospheric electricity.

However, because it was necessary to first verify the developed instruments on more deterministic processes than direct measuring during a storm activity, a large number of supportive experiments were conducted, which served to verify the properties of the instruments and their supporting subsystems. These experiments included flights of stratospheric balloons and UAVs as well as ground measurements with similar instrumentation either at stationary or mobile measuring stations. Chapter \ref[experiments] therefore documents the extensive preparatory work necessary to validate the proposed instruments' efficacy before their direct application in thunderstorm research.

Chapter \ref[results] synthesizes the insights gained from the application of  the developed instruments in the field, reflecting on the contributions this research has made to the field of atmospheric sciences, particularly in understanding lightning phenomena. It evaluates the effectiveness of the newly developed instrumentation in capturing comprehensive lightning data, discusses the impact of these findings on existing theories, and considers their practical implications. Furthermore, it outlines future research directions, suggesting areas where further investigation could yield substantial advancements in the study of atmospheric electricity with the additional use of similar instrumentation. This final section underscores the thesis' role in enhancing scientific knowledge and its potential for real-world applications in weather forecasting and safety measures and provides a detailed account of how the research objectives were achieved.
